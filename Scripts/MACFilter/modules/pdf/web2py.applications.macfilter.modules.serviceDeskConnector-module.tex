%
% API Documentation for Macfilter
% Module web2py.applications.macfilter.modules.serviceDeskConnector
%
% Generated by epydoc 3.0.1
% [Fri Oct 28 10:03:13 2011]
%

%%%%%%%%%%%%%%%%%%%%%%%%%%%%%%%%%%%%%%%%%%%%%%%%%%%%%%%%%%%%%%%%%%%%%%%%%%%
%%                          Module Description                           %%
%%%%%%%%%%%%%%%%%%%%%%%%%%%%%%%%%%%%%%%%%%%%%%%%%%%%%%%%%%%%%%%%%%%%%%%%%%%

    \index{web2py \textit{(package)}!web2py.applications \textit{(package)}!web2py.applications.macfilter \textit{(package)}!web2py.applications.macfilter.modules \textit{(package)}!web2py.applications.macfilter.modules.serviceDeskConnector \textit{(module)}|(}
\section{Module web2py.applications.macfilter.modules.serviceDeskConnector}

    \label{web2py:applications:macfilter:modules:serviceDeskConnector}
\textbf{Version:} 1.0



\textbf{Author:} Guido Accardo {\textless}gaccardo@coresecurity.com{\textgreater}



\textbf{Copyright:} Core Security Technologies



\textbf{License:} CORE




%%%%%%%%%%%%%%%%%%%%%%%%%%%%%%%%%%%%%%%%%%%%%%%%%%%%%%%%%%%%%%%%%%%%%%%%%%%
%%                               Variables                               %%
%%%%%%%%%%%%%%%%%%%%%%%%%%%%%%%%%%%%%%%%%%%%%%%%%%%%%%%%%%%%%%%%%%%%%%%%%%%

  \subsection{Variables}

    \vspace{-1cm}
\hspace{\varindent}\begin{longtable}{|p{\varnamewidth}|p{\vardescrwidth}|l}
\cline{1-2}
\cline{1-2} \centering \textbf{Name} & \centering \textbf{Description}& \\
\cline{1-2}
\endhead\cline{1-2}\multicolumn{3}{r}{\small\textit{continued on next page}}\\\endfoot\cline{1-2}
\endlastfoot\raggedright \_\-\_\-p\-a\-c\-k\-a\-g\-e\-\_\-\_\- & \raggedright \textbf{Value:} 
{\tt \texttt{'}\texttt{web2py.applications.macfilter.modules}\texttt{'}}&\\
\cline{1-2}
\end{longtable}


%%%%%%%%%%%%%%%%%%%%%%%%%%%%%%%%%%%%%%%%%%%%%%%%%%%%%%%%%%%%%%%%%%%%%%%%%%%
%%                           Class Description                           %%
%%%%%%%%%%%%%%%%%%%%%%%%%%%%%%%%%%%%%%%%%%%%%%%%%%%%%%%%%%%%%%%%%%%%%%%%%%%

    \index{web2py \textit{(package)}!web2py.applications \textit{(package)}!web2py.applications.macfilter \textit{(package)}!web2py.applications.macfilter.modules \textit{(package)}!web2py.applications.macfilter.modules.serviceDeskConnector \textit{(module)}!web2py.applications.macfilter.modules.serviceDeskConnector.serviceDeskConnector \textit{(class)}|(}
\subsection{Class serviceDeskConnector}

    \label{web2py:applications:macfilter:modules:serviceDeskConnector:serviceDeskConnector}
Conexion con la base de datos del service desk y guardar los datos validos 
en la base de datos locales


%%%%%%%%%%%%%%%%%%%%%%%%%%%%%%%%%%%%%%%%%%%%%%%%%%%%%%%%%%%%%%%%%%%%%%%%%%%
%%                                Methods                                %%
%%%%%%%%%%%%%%%%%%%%%%%%%%%%%%%%%%%%%%%%%%%%%%%%%%%%%%%%%%%%%%%%%%%%%%%%%%%

  \subsubsection{Methods}

    \label{web2py:applications:macfilter:modules:serviceDeskConnector:serviceDeskConnector:__init__}
    \index{web2py \textit{(package)}!web2py.applications \textit{(package)}!web2py.applications.macfilter \textit{(package)}!web2py.applications.macfilter.modules \textit{(package)}!web2py.applications.macfilter.modules.serviceDeskConnector \textit{(module)}!web2py.applications.macfilter.modules.serviceDeskConnector.serviceDeskConnector \textit{(class)}!web2py.applications.macfilter.modules.serviceDeskConnector.serviceDeskConnector.\_\_init\_\_ \textit{(method)}}

    \vspace{0.5ex}

\hspace{.8\funcindent}\begin{boxedminipage}{\funcwidth}

    \raggedright \textbf{\_\_init\_\_}(\textit{self})

\setlength{\parskip}{2ex}
\setlength{\parskip}{1ex}
    \end{boxedminipage}

    \label{web2py:applications:macfilter:modules:serviceDeskConnector:serviceDeskConnector:getLog}
    \index{web2py \textit{(package)}!web2py.applications \textit{(package)}!web2py.applications.macfilter \textit{(package)}!web2py.applications.macfilter.modules \textit{(package)}!web2py.applications.macfilter.modules.serviceDeskConnector \textit{(module)}!web2py.applications.macfilter.modules.serviceDeskConnector.serviceDeskConnector \textit{(class)}!web2py.applications.macfilter.modules.serviceDeskConnector.serviceDeskConnector.getLog \textit{(method)}}

    \vspace{0.5ex}

\hspace{.8\funcindent}\begin{boxedminipage}{\funcwidth}

    \raggedright \textbf{getLog}(\textit{self})

    \vspace{-1.5ex}

    \rule{\textwidth}{0.5\fboxrule}
\setlength{\parskip}{2ex}
    Obtener todos los logs hasta el momento

\setlength{\parskip}{1ex}
      \textbf{Return Value}
    \vspace{-1ex}

      \begin{quote}
      list

      \end{quote}

    \end{boxedminipage}

    \label{web2py:applications:macfilter:modules:serviceDeskConnector:serviceDeskConnector:doConnectionServiceDesk}
    \index{web2py \textit{(package)}!web2py.applications \textit{(package)}!web2py.applications.macfilter \textit{(package)}!web2py.applications.macfilter.modules \textit{(package)}!web2py.applications.macfilter.modules.serviceDeskConnector \textit{(module)}!web2py.applications.macfilter.modules.serviceDeskConnector.serviceDeskConnector \textit{(class)}!web2py.applications.macfilter.modules.serviceDeskConnector.serviceDeskConnector.doConnectionServiceDesk \textit{(method)}}

    \vspace{0.5ex}

\hspace{.8\funcindent}\begin{boxedminipage}{\funcwidth}

    \raggedright \textbf{doConnectionServiceDesk}(\textit{self})

    \vspace{-1.5ex}

    \rule{\textwidth}{0.5\fboxrule}
\setlength{\parskip}{2ex}
    Conectarse a la base de datos del service desk

\setlength{\parskip}{1ex}
      \textbf{Return Value}
    \vspace{-1ex}

      \begin{quote}
      MySQLdb.connection

      \end{quote}

    \end{boxedminipage}

    \label{web2py:applications:macfilter:modules:serviceDeskConnector:serviceDeskConnector:doConnectionMacfilter}
    \index{web2py \textit{(package)}!web2py.applications \textit{(package)}!web2py.applications.macfilter \textit{(package)}!web2py.applications.macfilter.modules \textit{(package)}!web2py.applications.macfilter.modules.serviceDeskConnector \textit{(module)}!web2py.applications.macfilter.modules.serviceDeskConnector.serviceDeskConnector \textit{(class)}!web2py.applications.macfilter.modules.serviceDeskConnector.serviceDeskConnector.doConnectionMacfilter \textit{(method)}}

    \vspace{0.5ex}

\hspace{.8\funcindent}\begin{boxedminipage}{\funcwidth}

    \raggedright \textbf{doConnectionMacfilter}(\textit{self})

    \vspace{-1.5ex}

    \rule{\textwidth}{0.5\fboxrule}
\setlength{\parskip}{2ex}
    Conectarse a la base de datos local

\setlength{\parskip}{1ex}
      \textbf{Return Value}
    \vspace{-1ex}

      \begin{quote}
      MySQLdb.connection

      \end{quote}

    \end{boxedminipage}

    \label{web2py:applications:macfilter:modules:serviceDeskConnector:serviceDeskConnector:getEsquemaId}
    \index{web2py \textit{(package)}!web2py.applications \textit{(package)}!web2py.applications.macfilter \textit{(package)}!web2py.applications.macfilter.modules \textit{(package)}!web2py.applications.macfilter.modules.serviceDeskConnector \textit{(module)}!web2py.applications.macfilter.modules.serviceDeskConnector.serviceDeskConnector \textit{(class)}!web2py.applications.macfilter.modules.serviceDeskConnector.serviceDeskConnector.getEsquemaId \textit{(method)}}

    \vspace{0.5ex}

\hspace{.8\funcindent}\begin{boxedminipage}{\funcwidth}

    \raggedright \textbf{getEsquemaId}(\textit{self}, \textit{name})

    \vspace{-1.5ex}

    \rule{\textwidth}{0.5\fboxrule}
\setlength{\parskip}{2ex}
    Dado el nombre de un esquema, devolver su id

\setlength{\parskip}{1ex}
      \textbf{Parameters}
      \vspace{-1ex}

      \begin{quote}
        \begin{Ventry}{xxxx}

          \item[name]

          Nombre del esquema a recuperar su id

            {\it (type=string)}

        \end{Ventry}

      \end{quote}

      \textbf{Return Value}
    \vspace{-1ex}

      \begin{quote}
      integer

      \end{quote}

    \end{boxedminipage}

    \label{web2py:applications:macfilter:modules:serviceDeskConnector:serviceDeskConnector:getVlanId}
    \index{web2py \textit{(package)}!web2py.applications \textit{(package)}!web2py.applications.macfilter \textit{(package)}!web2py.applications.macfilter.modules \textit{(package)}!web2py.applications.macfilter.modules.serviceDeskConnector \textit{(module)}!web2py.applications.macfilter.modules.serviceDeskConnector.serviceDeskConnector \textit{(class)}!web2py.applications.macfilter.modules.serviceDeskConnector.serviceDeskConnector.getVlanId \textit{(method)}}

    \vspace{0.5ex}

\hspace{.8\funcindent}\begin{boxedminipage}{\funcwidth}

    \raggedright \textbf{getVlanId}(\textit{self}, \textit{vlan}, \textit{esquema})

    \vspace{-1.5ex}

    \rule{\textwidth}{0.5\fboxrule}
\setlength{\parskip}{2ex}
    Dados el nombre de la vlan y el id del esquema, obtener el id de la 
    vlan

\setlength{\parskip}{1ex}
      \textbf{Parameters}
      \vspace{-1ex}

      \begin{quote}
        \begin{Ventry}{xxxxxxx}

          \item[vlan]

          nombre de la vlan

            {\it (type=string)}

          \item[esquema]

          id del esquema

            {\it (type=integer)}

        \end{Ventry}

      \end{quote}

      \textbf{Return Value}
    \vspace{-1ex}

      \begin{quote}
      integer

      \end{quote}

    \end{boxedminipage}

    \label{web2py:applications:macfilter:modules:serviceDeskConnector:serviceDeskConnector:getUsersMacfilter}
    \index{web2py \textit{(package)}!web2py.applications \textit{(package)}!web2py.applications.macfilter \textit{(package)}!web2py.applications.macfilter.modules \textit{(package)}!web2py.applications.macfilter.modules.serviceDeskConnector \textit{(module)}!web2py.applications.macfilter.modules.serviceDeskConnector.serviceDeskConnector \textit{(class)}!web2py.applications.macfilter.modules.serviceDeskConnector.serviceDeskConnector.getUsersMacfilter \textit{(method)}}

    \vspace{0.5ex}

\hspace{.8\funcindent}\begin{boxedminipage}{\funcwidth}

    \raggedright \textbf{getUsersMacfilter}(\textit{self})

    \vspace{-1.5ex}

    \rule{\textwidth}{0.5\fboxrule}
\setlength{\parskip}{2ex}
    Obtener todos los usuarios guardados en la base de datos local

\setlength{\parskip}{1ex}
      \textbf{Return Value}
    \vspace{-1ex}

      \begin{quote}
      list

      \end{quote}

    \end{boxedminipage}

    \label{web2py:applications:macfilter:modules:serviceDeskConnector:serviceDeskConnector:getUsersWithMac}
    \index{web2py \textit{(package)}!web2py.applications \textit{(package)}!web2py.applications.macfilter \textit{(package)}!web2py.applications.macfilter.modules \textit{(package)}!web2py.applications.macfilter.modules.serviceDeskConnector \textit{(module)}!web2py.applications.macfilter.modules.serviceDeskConnector.serviceDeskConnector \textit{(class)}!web2py.applications.macfilter.modules.serviceDeskConnector.serviceDeskConnector.getUsersWithMac \textit{(method)}}

    \vspace{0.5ex}

\hspace{.8\funcindent}\begin{boxedminipage}{\funcwidth}

    \raggedright \textbf{getUsersWithMac}(\textit{self})

    \vspace{-1.5ex}

    \rule{\textwidth}{0.5\fboxrule}
\setlength{\parskip}{2ex}
    Obtenet todos los usuarios del service desk que tengan al menos una mac
    address en su perfil

\setlength{\parskip}{1ex}
      \textbf{Return Value}
    \vspace{-1ex}

      \begin{quote}
      list

      \end{quote}

    \end{boxedminipage}

    \label{web2py:applications:macfilter:modules:serviceDeskConnector:serviceDeskConnector:evalUsersToAdd}
    \index{web2py \textit{(package)}!web2py.applications \textit{(package)}!web2py.applications.macfilter \textit{(package)}!web2py.applications.macfilter.modules \textit{(package)}!web2py.applications.macfilter.modules.serviceDeskConnector \textit{(module)}!web2py.applications.macfilter.modules.serviceDeskConnector.serviceDeskConnector \textit{(class)}!web2py.applications.macfilter.modules.serviceDeskConnector.serviceDeskConnector.evalUsersToAdd \textit{(method)}}

    \vspace{0.5ex}

\hspace{.8\funcindent}\begin{boxedminipage}{\funcwidth}

    \raggedright \textbf{evalUsersToAdd}(\textit{self})

    \vspace{-1.5ex}

    \rule{\textwidth}{0.5\fboxrule}
\setlength{\parskip}{2ex}
    Analiza si los usuario deben ser agregados y a que vlan y los devuelve

\setlength{\parskip}{1ex}
      \textbf{Return Value}
    \vspace{-1ex}

      \begin{quote}
      list

      \end{quote}

    \end{boxedminipage}

    \label{web2py:applications:macfilter:modules:serviceDeskConnector:serviceDeskConnector:addUsersToMacfilter}
    \index{web2py \textit{(package)}!web2py.applications \textit{(package)}!web2py.applications.macfilter \textit{(package)}!web2py.applications.macfilter.modules \textit{(package)}!web2py.applications.macfilter.modules.serviceDeskConnector \textit{(module)}!web2py.applications.macfilter.modules.serviceDeskConnector.serviceDeskConnector \textit{(class)}!web2py.applications.macfilter.modules.serviceDeskConnector.serviceDeskConnector.addUsersToMacfilter \textit{(method)}}

    \vspace{0.5ex}

\hspace{.8\funcindent}\begin{boxedminipage}{\funcwidth}

    \raggedright \textbf{addUsersToMacfilter}(\textit{self}, \textit{sync}={\tt None})

    \vspace{-1.5ex}

    \rule{\textwidth}{0.5\fboxrule}
\setlength{\parskip}{2ex}
    Luego de evaluar que movimientos de usuarios realizar, los realiza.

\setlength{\parskip}{1ex}
    \end{boxedminipage}

    \label{web2py:applications:macfilter:modules:serviceDeskConnector:serviceDeskConnector:deleteUsersNotInHD}
    \index{web2py \textit{(package)}!web2py.applications \textit{(package)}!web2py.applications.macfilter \textit{(package)}!web2py.applications.macfilter.modules \textit{(package)}!web2py.applications.macfilter.modules.serviceDeskConnector \textit{(module)}!web2py.applications.macfilter.modules.serviceDeskConnector.serviceDeskConnector \textit{(class)}!web2py.applications.macfilter.modules.serviceDeskConnector.serviceDeskConnector.deleteUsersNotInHD \textit{(method)}}

    \vspace{0.5ex}

\hspace{.8\funcindent}\begin{boxedminipage}{\funcwidth}

    \raggedright \textbf{deleteUsersNotInHD}(\textit{self})

    \vspace{-1.5ex}

    \rule{\textwidth}{0.5\fboxrule}
\setlength{\parskip}{2ex}
    Borrar todos los usuario para volver a crearlos

\setlength{\parskip}{1ex}
    \end{boxedminipage}

    \index{web2py \textit{(package)}!web2py.applications \textit{(package)}!web2py.applications.macfilter \textit{(package)}!web2py.applications.macfilter.modules \textit{(package)}!web2py.applications.macfilter.modules.serviceDeskConnector \textit{(module)}!web2py.applications.macfilter.modules.serviceDeskConnector.serviceDeskConnector \textit{(class)}|)}
    \index{web2py \textit{(package)}!web2py.applications \textit{(package)}!web2py.applications.macfilter \textit{(package)}!web2py.applications.macfilter.modules \textit{(package)}!web2py.applications.macfilter.modules.serviceDeskConnector \textit{(module)}|)}
